\documentclass[a4paper,12pt]{report}
\usepackage[utf8]{inputenc}
\usepackage[T1]{fontenc}
\usepackage[french]{babel}
\usepackage{graphicx}
\usepackage{subcaption}
\usepackage{caption}
\usepackage{titlesec}
\usepackage{geometry}
\usepackage{tikz}
\usepackage{ragged2e}
\usetikzlibrary{shapes.geometric, arrows.meta, shadows, positioning}
\usepackage{lmodern}
\usepackage{tocloft}
\usepackage{float}
\usepackage{array}
\usepackage[table]{xcolor}
\usepackage{listings}
\usepackage{hyperref}
\usepackage{fancyhdr}
\usepackage{framed}
\usepackage{mdframed}
\usepackage{enumitem}
\usepackage{booktabs}

% Marges et géométrie
\geometry{left=2.5cm, right=2.5cm, top=2.5cm, bottom=2.5cm, headheight=15pt}

% Configuration des hyperliens
\hypersetup{
    colorlinks=true,
    linkcolor=blue!50!black,
    filecolor=magenta,
    urlcolor=cyan!70!black,
    citecolor=green!50!black,
    pdftitle={Système de Gestion des Réservations d'Hôtel},
    pdfauthor={Yazid TAHRI ALAOUI, Dounia LAMDAKKI, Kaoutar BENOUAHI},
    pdfsubject={Structure de Données et Algorithmique Avancées},
    pdfkeywords={C, Hôtel, Réservation, ncurses, ENSAH}
}

% En-têtes et pieds de page personnalisés
\pagestyle{fancy}
\fancyhf{}
\fancyhead[L]{\small\textit{Système de Gestion des Réservations d'Hôtel}}
\fancyhead[R]{\small\textit{ENSAH - ID1 2024/2025}}
\fancyfoot[C]{\thepage}
\renewcommand{\headrulewidth}{0.5pt}
\renewcommand{\footrulewidth}{0.5pt}

% Style pour la page de garde et chapitres
\fancypagestyle{plain}{
    \fancyhf{}
    \fancyfoot[C]{\thepage}
    \renewcommand{\headrulewidth}{0pt}
    \renewcommand{\footrulewidth}{0.5pt}
}

% Configuration des couleurs personnalisées
\definecolor{ensahblue}{RGB}{0, 51, 102}
\definecolor{ensahgold}{RGB}{204, 153, 0}
\definecolor{codebackground}{RGB}{248, 248, 248}
\definecolor{commentgreen}{RGB}{34, 139, 34}

% Configuration améliorée pour les listings de code
\lstset{
    basicstyle=\ttfamily\small,
    keywordstyle=\color{blue!80!black}\bfseries,
    commentstyle=\color{commentgreen}\itshape,
    stringstyle=\color{red!70!black},
    showstringspaces=false,
    breaklines=true,
    frame=single,
    frameround=tttt,
    numbers=left,
    numberstyle=\tiny\color{gray},
    numbersep=8pt,
    language=C,
    backgroundcolor=\color{codebackground},
    rulecolor=\color{ensahblue},
    captionpos=b,
    aboveskip=1.5em,
    belowskip=1em,
    xleftmargin=2em,
    xrightmargin=1em,
    framexleftmargin=1.5em,
    tabsize=4
}

% Configuration des figures avec bordures
\captionsetup[figure]{
    format=hang,
    font={small,it},
    labelfont={bf,color=ensahblue},
    justification=centering,
    skip=10pt
}

% Configuration des tableaux
\captionsetup[table]{
    format=hang,
    font={small,it},
    labelfont={bf,color=ensahblue},
    justification=centering,
    skip=10pt,
    position=top
}

% Environnement pour les figures avec bordure
\newenvironment{borderedfigure}
    {\begin{mdframed}[linecolor=ensahblue, linewidth=1.5pt, roundcorner=5pt, 
        innertopmargin=10pt, innerbottommargin=10pt, 
        innerleftmargin=10pt, innerrightmargin=10pt,
        backgroundcolor=white]}
    {\end{mdframed}}

% Format des titres de chapitre - configuration simplifiée
\titleformat{\chapter}[display]
{\normalfont\huge\bfseries\color{ensahblue}}
{\chaptertitlename\ \thechapter}{20pt}{\Huge}

% Format des sections
\titleformat{\section}
{\normalfont\Large\bfseries\color{ensahblue}}
{\thesection}{1em}{}

% Format des subsections
\titleformat{\subsection}
{\normalfont\large\bfseries\color{ensahblue!80!black}}
{\thesubsection}{1em}{}

% Configuration des listes
\setlist[itemize]{leftmargin=*,label=\textcolor{ensahblue}{$\bullet$}}
\setlist[enumerate]{leftmargin=*}

% Configuration de la table des matières
\renewcommand{\cftchapfont}{\bfseries\color{ensahblue}}
\renewcommand{\cftsecfont}{\color{black}}
\renewcommand{\cftsubsecfont}{\color{black!70}}

% Amélioration des tableaux
\newcolumntype{L}[1]{>{\raggedright\arraybackslash}p{#1}}
\newcolumntype{C}[1]{>{\centering\arraybackslash}p{#1}}
\newcolumntype{R}[1]{>{\raggedleft\arraybackslash}p{#1}}

\begin{document}

% ---------- Page de garde ----------
\begin{titlepage}
    \begin{minipage}{0.3\textwidth}
        \includegraphics[width=3.5cm]{logos/UAE.png}
    \end{minipage}
    \hfill
    \begin{minipage}{0.3\textwidth}
        \raggedleft
        \includegraphics[width=3.5cm]{logos/ENSAH.png}
    \end{minipage}
    
    \vspace*{0.2cm}
    
    \begin{center}
        {\large \textbf{ROYAUME DU MAROC}}\\[0.3cm]
        {\large \textbf{Université Abdelmalek Essaâdi}}\\[0.3cm]
        {\large \textbf{École Nationale des Sciences Appliquées d'Al Hoceima}}\\[2cm]
        
        \rule{1\linewidth}{0.5pt}\\[0.2cm]
        \huge{\textbf{Rapport de Projet}}\\[0.3cm]
        \huge{\textbf{Système de Gestion des Réservations d'Hôtel}}\\[0.2cm]
        \rule{0.6\linewidth}{0.5pt}\\[0.2cm]
        {\large \textbf{Structure de Données et Algorithmique Avancées}}\\[0.2cm]
    
        \vspace{2cm}
        \large\textbf{Réalisé par :}\\[0.2cm]
        Yazid TAHRI ALAOUI\\[0.1cm]
        Dounia LAMDAKKI\\[0.1cm]
        Kaoutar BENOUAHI\\[0.4cm]
        \large\textbf{Filière :} 1ère année – Ingénierie Données (ID1)\\[0.2cm]
        \large \textbf{Encadré par :} Dr. BAHRI Abdelkhalek\\
        
        \vspace*{2cm}
        {\large \textbf{Année universitaire : 2024 -- 2025}}
    \end{center}
    
    \vfill
\end{titlepage}

\newgeometry{left=2.5cm, right=2.5cm, top=1.5cm, bottom=2.5cm}

% ---------- Numérotation en chiffres romains ----------
\renewcommand{\thechapter}{\Roman{chapter}}

% ---------- Remerciements ----------
\chapter*{Remerciements}
\addcontentsline{toc}{chapter}{Remerciements}
\begin{center}
{\large
La réalisation de ce projet ne saurait être réduite à une simple accumulation de lignes de code ; elle est le fruit d'une synergie, d'un apprentissage continu et d'un accompagnement précieux. Avant d'entamer la présentation technique de notre application de gestion hôtelière, il nous est agréable de témoigner notre reconnaissance envers ceux qui ont contribué, de près ou de loin, à l'aboutissement de ce travail.\\[0.4cm]

Nous tenons en premier lieu à exprimer notre profonde gratitude à notre professeur, Monsieur \textbf{Dr. BAHRI Abdelkhalek}. Au-delà des connaissances techniques en langage C qu'il a su nous transmettre avec clarté, nous le remercions pour sa pédagogie exigeante mais bienveillante. Ses directives précises et son insistance sur la rigueur architecturale nous ont poussés à ne pas nous contenter de solutions fonctionnelles, mais à rechercher l'optimisation et la propreté du code. Merci de nous avoir guidés dans le labyrinthe de la gestion de la mémoire et des pointeurs, transformant nos difficultés en opportunités d'apprentissage.\\[0.4cm]

Nos remerciements s'adressent également au corps professoral et administratif de l'École Nationale des Sciences Appliquées d'Al Hoceima, pour la qualité de l'environnement d'étude et les ressources mises à notre disposition, propices à la recherche et à l'innovation.\\[0.4cm]

Nous n'oublions pas nos familles et nos proches, dont le soutien moral inconditionnel et les encouragements constants ont été notre moteur, particulièrement lors des phases de débogage les plus ardues et des nuits blanches consacrées à la finalisation du projet.\\[0.4cm]

Enfin, une pensée particulière pour nos collègues de promotion, avec qui les échanges fructueux et l'entraide ont permis de surmonter les obstacles techniques et de maintenir une motivation intacte tout au long de ce semestre.\\[0.4cm]

Que tous trouvent ici l'expression de notre estime et de notre respect.
\vfill
}
\end{center}

% ---------- Résumé ----------
\chapter*{Résumé}
\addcontentsline{toc}{chapter}{Résumé}
\begin{center}
{\large
L'industrie hôtelière moderne nécessite des solutions informatiques robustes pour gérer efficacement les opérations quotidiennes. Ce rapport présente la conception et l'implémentation d'un système complet de gestion hôtelière développé en langage C, utilisant la bibliothèque ncurses pour offrir une interface utilisateur textuelle riche et intuitive.\\[0.4cm]

Notre application propose une solution modulaire intégrant la gestion des clients, l'inventaire des chambres, le système de réservations et la facturation automatisée. Le système implémente un contrôle d'accès basé sur les rôles (RBAC) avec trois niveaux de privilèges : administrateur, réceptionniste et client, garantissant la sécurité et la séparation des responsabilités.\\[0.4cm]

L'architecture du projet repose sur des structures de données optimisées et des algorithmes efficaces pour la recherche, le tri et la validation des données. La persistance des informations est assurée par un système de fichiers binaires, permettant la sauvegarde et la récupération des données entre les sessions.\\[0.4cm]

Ce travail démontre notre maîtrise des concepts avancés de programmation en C, incluant la gestion dynamique de la mémoire, la manipulation de pointeurs, la programmation modulaire et l'utilisation de bibliothèques système. L'application résultante offre une expérience utilisateur professionnelle comparable aux solutions commerciales, tout en restant légère et performante.
\vfill
}
\end{center}

% ---------- Abstract ----------
\chapter*{Abstract}
\addcontentsline{toc}{chapter}{Abstract}
\begin{center}
{\large
The modern hotel industry requires robust IT solutions to efficiently manage daily operations. This report presents the design and implementation of a comprehensive hotel management system developed in C language, using the ncurses library to provide a rich and intuitive text-based user interface.\\[0.4cm]

Our application offers a modular solution integrating client management, room inventory, reservation system, and automated billing. The system implements role-based access control (RBAC) with three privilege levels: administrator, receptionist, and client, ensuring security and separation of responsibilities.\\[0.4cm]

The project architecture is based on optimized data structures and efficient algorithms for searching, sorting, and data validation. Data persistence is ensured by a binary file system, allowing data storage and retrieval between sessions.\\[0.4cm]

This work demonstrates our mastery of advanced C programming concepts, including dynamic memory management, pointer manipulation, modular programming, and system library usage. The resulting application provides a professional user experience comparable to commercial solutions while remaining lightweight and performant.
\vfill
}
\end{center}

% ---------- Table des matières ----------
\newpage
\renewcommand{\contentsname}{Table des matières}
\tableofcontents
\newpage

% Changer le titre de la table des figures
\renewcommand{\listfigurename}{Liste des figures}
\renewcommand{\cftfigpresnum}{Figure~}
\renewcommand{\cftfigaftersnum}{:}
\renewcommand{\cftfignumwidth}{2.2cm}
\setlength{\cftbeforefigskip}{1pt}
\setlength{\cftaftertoctitleskip}{1em}

\listoffigures
\newpage

% ---------- Retour à la numérotation arabe ----------
\renewcommand{\thechapter}{\arabic{chapter}}
\setcounter{chapter}{0}

% ========== CHAPITRE 1 : INTRODUCTION ==========
\chapter{Introduction Générale}

\section{Contexte du Projet}

L'industrie hôtelière est un secteur où la qualité de service et la réactivité sont des vecteurs essentiels de compétitivité. Dans ce contexte dynamique, la gestion de l'information — qu'il s'agisse des disponibilités des chambres, des préférences des clients ou des transactions financières — ne peut plus se permettre d'être approximative. La transition numérique s'impose donc non plus comme une option, mais comme une nécessité opérationnelle pour toute structure souhaitant optimiser son rendement et garantir la satisfaction de sa clientèle.

C'est dans ce cadre que s'inscrit notre projet académique : la conception d'un système d'information complet, développé en langage C, capable de piloter les activités fondamentales d'un établissement hôtelier.

\section{Problématique}

Malgré l'avènement des technologies numériques, de nombreuses structures modestes continuent de s'appuyer sur des méthodes de gestion artisanales. L'utilisation de registres manuscrits ou de tableurs non centralisés engendre des dysfonctionnements critiques :

\begin{itemize}
    \item \textbf{Perte d'intégrité des données} : Risque élevé d'erreurs humaines lors de la saisie ou de la recopie d'informations (nom mal orthographié, dates erronées).
    \item \textbf{Manque de visibilité temps réel} : Impossibilité pour le gérant de connaître instantanément le taux d'occupation ou les chambres nécessitant un nettoyage.
    \item \textbf{Insécurité des informations} : Absence de contrôle d'accès aux données sensibles des clients et volatilité des supports physiques.
    \item \textbf{Conflits de réservation} : Le phénomène de surréservation (overbooking) est fréquent lorsque la mise à jour des plannings n'est pas automatisée et synchronisée.
\end{itemize}

Dès lors, la question centrale de ce projet est la suivante : \textit{Comment concevoir une architecture logicielle modulaire et robuste en langage C, capable d'automatiser le flux de travail hôtelier tout en garantissant la persistance et la sécurité des données ?}

\section{Objectifs du Projet}

Ce projet poursuit un double objectif, à la fois fonctionnel et pédagogique :

\subsection{Objectifs Fonctionnels}

L'application vise à fournir une interface console ergonomique permettant :

\begin{itemize}
    \item Une gestion centralisée et sécurisée du parc immobilier (chambres) et du fichier clientèle.
    \item L'automatisation du cycle de vie d'une réservation, de la vérification des disponibilités à la facturation.
    \item La pérennité des informations entre les sessions d'utilisation grâce à un système de sauvegarde sur fichiers binaires.
    \item Un contrôle d'accès basé sur les rôles pour garantir la sécurité et la séparation des privilèges.
\end{itemize}

\subsection{Objectifs Techniques}

Sur le plan technique, ce travail est l'occasion de démontrer notre maîtrise des concepts avancés du langage C, notamment :

\begin{itemize}
    \item La programmation modulaire pour structurer un code complexe.
    \item La manipulation de fichiers binaires pour la persistance des données.
    \item L'utilisation de structures de données et de pointeurs pour optimiser la gestion de la mémoire.
    \item La mise en œuvre d'algorithmes de recherche et de validation pour l'exploitation des données.
    \item L'intégration de la bibliothèque ncurses pour créer une interface utilisateur professionnelle.
\end{itemize}

\section{Annonce du Plan}

Pour rendre compte de notre démarche, ce rapport s'articulera autour de quatre axes majeurs :

\begin{enumerate}
    \item \textbf{Analyse et Conception} : Nous détaillerons l'architecture des données, les choix techniques et la modélisation du système.
    \item \textbf{Réalisation Technique} : Nous expliquerons l'implémentation des modules clés, les algorithmes développés et les défis techniques rencontrés.
    \item \textbf{Validation et Tests} : Nous présenterons le guide d'utilisation à travers des scénarios de test et des captures d'écran.
    \item \textbf{Conclusion} : Nous conclurons sur les acquis et les perspectives d'évolution de l'application.
\end{enumerate}

% ========== CHAPITRE 2 : ANALYSE ET CONCEPTION ==========
\chapter{Analyse et Conception du Système}

\section{Architecture Générale}

Notre système de gestion hôtelière repose sur une architecture modulaire en trois couches distinctes, garantissant la séparation des responsabilités et la maintenabilité du code :

\begin{figure}[H]
    \centering
    \begin{tikzpicture}[node distance=2cm]
        \tikzstyle{layer} = [rectangle, minimum width=8cm, minimum height=1.5cm, text centered, draw=black, fill=blue!20, text width=7.5cm, align=center]
        \tikzstyle{arrow} = [thick,->,>=stealth]
        
        \node (ui) [layer] {\textbf{Couche Présentation (UI)}\\Interface ncurses, gestion des événements};
        \node (logic) [layer, below of=ui] {\textbf{Couche Métier}\\Gestion clients, chambres, réservations};
        \node (data) [layer, below of=logic] {\textbf{Couche Données}\\Persistance fichiers binaires};
        
        \draw [arrow] (ui) -- (logic);
        \draw [arrow] (logic) -- (data);
    \end{tikzpicture}
    \caption{Architecture en trois couches du système}
\end{figure}

\subsection{Couche Présentation}

La couche présentation utilise la bibliothèque \texttt{ncurses} pour offrir une interface utilisateur textuelle riche. Elle comprend :

\begin{itemize}
    \item \textbf{Système de thèmes} : Gestion des couleurs et du style visuel
    \item \textbf{Composants réutilisables} : Formulaires, listes, boîtes de dialogue
    \item \textbf{Gestion des événements} : Navigation clavier, validation des saisies
    \item \textbf{Routage des vues} : Machine à états pour la navigation entre écrans
\end{itemize}

\subsection{Couche Métier}

La couche métier implémente la logique applicative :

\begin{itemize}
    \item \textbf{Module Clients} : CRUD des informations client
    \item \textbf{Module Chambres} : Gestion de l'inventaire et disponibilités
    \item \textbf{Module Réservations} : Validation des dates, calcul des nuitées
    \item \textbf{Module Facturation} : Génération automatique des factures
    \item \textbf{Module Authentification} : Gestion des utilisateurs et RBAC
\end{itemize}

\subsection{Couche Données}

La couche données assure la persistance via des fichiers binaires :

\begin{itemize}
    \item \texttt{clients.dat} : Enregistrements des clients
    \item \texttt{chambres.dat} : Inventaire des chambres
    \item \texttt{reservations.dat} : Historique des réservations
    \item \texttt{factures.dat} : Archive des factures
    \item \texttt{users.dat} : Base de données des utilisateurs
\end{itemize}

\section{Modélisation des Données}

\subsection{Structures de Données Principales}

Notre système utilise quatre structures de données fondamentales, définies dans \texttt{structures.h} :

\begin{lstlisting}[caption={Structure Client}]
typedef struct {
    int id;
    char nom[50];
    char prenom[50];
    char email[100];
    char telephone[20];
} Client;
\end{lstlisting}

\begin{lstlisting}[caption={Structure Chambre}]
typedef struct {
    int numero;
    char type[20];
    float prix;
    int disponible;
} Chambre;
\end{lstlisting}

\begin{lstlisting}[caption={Structure Reservation}]
typedef struct {
    int id;
    int client_id;
    int chambre_numero;
    char date_debut[11];
    char date_fin[11];
    float montant;
    char statut[20];
} Reservation;
\end{lstlisting}

\begin{lstlisting}[caption={Structure Facture}]
typedef struct {
    int idFacture;
    int idClient;
    int nbNuits;
    float prixNuit;
    float total;
} Facture;
\end{lstlisting}

\subsection{Diagramme de Relations}

\begin{figure}[H]
    \centering
    \begin{tikzpicture}[node distance=3cm]
        \tikzstyle{entity} = [rectangle, minimum width=3cm, minimum height=1cm, text centered, draw=black, fill=orange!30, text width=2.8cm, align=center]
        \tikzstyle{arrow} = [thick,->,>=stealth]
        
        \node (client) [entity] {\textbf{Client}\\id, nom, email};
        \node (reservation) [entity, right of=client, xshift=3cm] {\textbf{Réservation}\\id, dates, montant};
        \node (chambre) [entity, below of=reservation] {\textbf{Chambre}\\numero, type, prix};
        \node (facture) [entity, below of=client] {\textbf{Facture}\\id, total};
        
        \draw [arrow] (client) -- node[anchor=south] {1:N} (reservation);
        \draw [arrow] (chambre) -- node[anchor=west] {1:N} (reservation);
        \draw [arrow] (reservation) -- node[anchor=east] {1:1} (facture);
    \end{tikzpicture}
    \caption{Diagramme des relations entre entités}
\end{figure}

\section{Diagrammes UML du Système}

Cette section présente les diagrammes UML créés lors de la phase de conception du projet. Ces diagrammes formalisent l'architecture, les interactions et les comportements du système de gestion hôtelière selon les standards de modélisation UML 2.0.

\subsection{Diagramme de Cas d'Utilisation}

Le diagramme de cas d'utilisation identifie les acteurs du système et leurs interactions avec les fonctionnalités de l'application.

\begin{figure}[H]
    \centering
    \includegraphics[width=0.9\textwidth]{diagramms/S-PROJET/SA (2).png}
    \caption{Diagramme de cas d'utilisation global du système}
\end{figure}

Ce diagramme illustre les trois acteurs principaux :
\begin{itemize}
    \item \textbf{Administrateur} : Dispose de tous les privilèges, incluant la gestion des utilisateurs, la supervision complète du système et l'accès aux fonctionnalités avancées de reporting.
    \item \textbf{Réceptionniste} : Gère les opérations quotidiennes de l'hôtel (enregistrement des clients, gestion des réservations, facturation).
    \item \textbf{Client} : Utilise le portail libre-service pour consulter les disponibilités, effectuer des réservations et visualiser son historique.
\end{itemize}

Les cas d'utilisation principaux incluent :
\begin{itemize}
    \item Gestion de l'authentification et des sessions
    \item CRUD complet sur les entités (Clients, Chambres, Réservations)
    \item Vérification de disponibilité en temps réel
    \item Génération et export de factures
    \item Consultation d'historique et statistiques
\end{itemize}

\subsection{Diagramme de Classes}

Le diagramme de classes représente la structure statique du système, définissant les entités, leurs attributs et leurs relations.

\begin{figure}[H]
    \centering
    \includegraphics[width=0.85\textwidth]{diagramms/S-PROJET/SB (2).png}
    \caption{Diagramme de classes du système de réservation}
\end{figure}

Les classes principales sont :
\begin{itemize}
    \item \textbf{Client} : Encapsule les informations personnelles (id, nom, prénom, email, téléphone). Relation 1:N avec Réservation.
    \item \textbf{Chambre} : Représente l'inventaire (numéro, type, prix, disponibilité). Relation 1:N avec Réservation.
    \item \textbf{Réservation} : Classe associative reliant Client et Chambre, contenant les dates et le montant. Relation 1:1 avec Facture.
    \item \textbf{Facture} : Génère les documents de facturation (ID, nombre de nuits, total calculé).
    \item \textbf{Utilisateur} : Gère l'authentification (username, mot de passe hashé, rôle).
\end{itemize}

Les associations sont implémentées via des clés étrangères dans les structures C, respectant les contraintes d'intégrité référentielle.

\subsection{Diagrammes de Séquence}

Les diagrammes de séquence détaillent les interactions temporelles entre objets lors de l'exécution des cas d'utilisation critiques.

\subsubsection{Séquence : Authentification Utilisateur}

\begin{figure}[H]
    \centering
    \includegraphics[width=0.9\textwidth]{diagramms/S-PROJET/SC.png}
    \caption{Diagramme de séquence - Authentification et routage}
\end{figure}

Ce diagramme illustre le flux d'authentification :
\begin{enumerate}
    \item L'utilisateur saisit ses identifiants dans l'interface de login
    \item Le module d'authentification valide les credentials contre la base \texttt{users.dat}
    \item En cas de succès, le système vérifie le rôle de l'utilisateur
    \item Routage vers le dashboard approprié (Admin, Réceptionniste ou Client)
    \item Initialisation de la session avec chargement des données pertinentes
\end{enumerate}

La séparation stricte des flux selon les rôles garantit le respect du principe de moindre privilège.

\subsubsection{Séquence : Création de Réservation}

\begin{figure}[H]
    \centering
    \includegraphics[width=0.9\textwidth]{diagramms/S-PROJET/SD.png}
    \caption{Diagramme de séquence - Processus de réservation}
\end{figure}

Le processus de réservation suit un workflow strict :
\begin{enumerate}
    \item Sélection du client (recherche ou création)
    \item Choix de la chambre avec affichage des disponibilités
    \item Saisie des dates avec validation immédiate (format, cohérence, dates futures)
    \item Vérification des conflits de réservation via l'algorithme de chevauchement
    \item Calcul automatique du montant (nuitées × prix/nuit)
    \item Confirmation et persistance dans \texttt{reservations.dat}
    \item Mise à jour du statut de disponibilité de la chambre
\end{enumerate}

Cette séquence garantit l'intégrité des données et prévient les surréservations.

\subsubsection{Séquence : Génération de Facture}

\begin{figure}[H]
    \centering
    \includegraphics[width=0.9\textwidth]{diagramms/S-PROJET/SE.png}
    \caption{Diagramme de séquence - Facturation automatisée}
\end{figure}

Le module de facturation automatise les calculs :
\begin{enumerate}
    \item Sélection d'une réservation active
    \item Récupération des informations client et chambre
    \item Calcul précis des nuitées avec \texttt{mktime()}/\texttt{difftime()}
    \item Génération d'un ID unique pour la facture
    \item Création de l'objet Facture avec tous les détails
    \item Sauvegarde dans \texttt{factures.dat}
    \item Export optionnel au format texte (\texttt{facture\_N.txt})
\end{enumerate}

\subsection{Diagramme d'Activité}

Les diagrammes d'activité modélisent les workflows métier et les logiques de contrôle complexes.

\begin{figure}[H]
    \centering
    \includegraphics[width=0.8\textwidth]{diagramms/S-PROJET/SF.png}
    \caption{Diagramme d'activité - Workflow de réservation}
\end{figure}

Ce diagramme présente le flux complet d'une réservation, incluant :
\begin{itemize}
    \item Les points de décision (dates valides ? chambre disponible ?)
    \item Les actions parallèles (validation multiples critères)
    \item Les chemins alternatifs (succès vs échec)
    \item Les actions post-traitement (notification, logging)
\end{itemize}

Le parallélisme conceptuel est implémenté via des appels de fonctions de validation séquentiels en C.

\subsection{Diagramme d'États-Transitions}

Le diagramme d'états modélise les cycles de vie des entités principales.

\begin{figure}[H]
    \centering
    \includegraphics[width=0.85\textwidth]{diagramms/S-PROJET/SG.png}
    \caption{Diagramme d'états - Cycle de vie d'une réservation}
\end{figure}

Une réservation transite par les états suivants :
\begin{itemize}
    \item \textbf{CRÉÉE} : État initial après validation et persistance
    \item \textbf{ACTIVE} : Réservation confirmée et effective
    \item \textbf{EN\_COURS} : Client actuellement à l'hôtel (check-in effectué)
    \item \textbf{TERMINÉE} : Séjour achevé, facture générée
    \item \textbf{ANNULÉE} : Réservation annulée (soft delete)
\end{itemize}

Les transitions sont déclenchées par des événements métier (check-in, check-out, annulation) et respectent les contraintes temporelles.

\subsection{Architecture des Composants}

\begin{figure}[H]
    \centering
    \includegraphics[width=0.9\textwidth]{diagramms/S-PROJET/SH.png}
    \caption{Diagramme de composants - Architecture modulaire}
\end{figure}

Ce diagramme présente la décomposition modulaire du système :
\begin{itemize}
    \item \textbf{Composant UI} : Modules ncurses (\texttt{ui\_*.c}) gérant l'affichage et les interactions
    \item \textbf{Composant Métier} : Logique applicative (\texttt{clients.c}, \texttt{chambres.c}, \texttt{reservations.c})
    \item \textbf{Composant Données} : Couche de persistance (\texttt{fichiers.c}) avec abstraction des I/O
    \item \textbf{Composant Sécurité} : Module d'authentification (\texttt{auth.c}) et gestion RBAC
\end{itemize}

Les interfaces entre composants sont clairement définies via les fichiers d'en-tête (\texttt{.h}), favorisant le découplage et la réutilisabilité.

\subsection{Diagramme de Déploiement}

\begin{figure}[H]
    \centering
    \includegraphics[width=0.8\textwidth]{diagramms/S-PROJET/SI.png}
    \caption{Diagramme de déploiement - Architecture physique}
\end{figure}

Le déploiement illustre l'architecture mono-poste :
\begin{itemize}
    \item \textbf{Nœud Applicatif} : Exécutable compilé (\texttt{hotel\_app.exe})
    \item \textbf{Stockage Local} : Répertoire \texttt{data/} contenant les fichiers binaires
    \item \textbf{Dépendances} : Bibliothèque ncurses, runtime GCC
    \item \textbf{Terminal} : Interface d'interaction utilisateur
\end{itemize}

Cette architecture simplifie le déploiement (copie de répertoire) et garantit la portabilité entre systèmes compatibles POSIX.

\subsection{Vue d'Ensemble du Système}

\begin{figure}[H]
    \centering
    \includegraphics[width=0.95\textwidth]{diagramms/S-PROJET/SJ.jpg}
    \caption{Architecture globale du système de gestion hôtelière}
\end{figure}

Cette vue synthétique intègre tous les aspects du système :
\begin{itemize}
    \item Les acteurs et leurs niveaux d'accès
    \item Les modules fonctionnels et leurs interactions
    \item Les flux de données entre composants
    \item Les mécanismes de persistance et de sécurité
\end{itemize}

Cette modélisation complète a guidé l'implémentation et servi de référence lors des phases de développement et de débogage.

\section{Système de Contrôle d'Accès}

\subsection{Modèle RBAC}

Notre application implémente un contrôle d'accès basé sur les rôles (Role-Based Access Control) avec trois profils utilisateur :

\begin{table}[H]
\centering
\rowcolors{2}{gray!10}{white}
\begin{tabular}{L{3.5cm}L{11cm}}
\toprule
\textbf{\color{ensahblue}Rôle} & \textbf{\color{ensahblue}Privilèges} \\
\midrule
\textbf{Administrateur} & Accès complet : gestion utilisateurs, clients, chambres, réservations, facturation, configuration système \\
\addlinespace[0.2em]
\textbf{Réceptionniste} & Gestion clients, chambres, réservations, facturation (sans gestion utilisateurs ni configuration) \\
\addlinespace[0.2em]
\textbf{Client} & Portail libre-service : consultation disponibilités, création et gestion de réservations personnelles \\
\bottomrule
\end{tabular}
\caption{Matrice des privilèges par rôle (RBAC)}
\end{table}

\subsection{Flux d'Authentification}

\begin{figure}[H]
    \centering
    \begin{tikzpicture}[node distance=2.5cm, auto]
        \tikzstyle{process} = [rectangle, minimum width=2.5cm, minimum height=1cm, text centered, draw=black, fill=green!20, text width=2.3cm, align=center]
        \tikzstyle{decision} = [diamond, minimum width=2cm, minimum height=1cm, text centered, draw=black, fill=yellow!20, aspect=2, text width=1.8cm, align=center]
        \tikzstyle{arrow} = [thick,->,>=stealth]
        
        \node (start) [process] {Écran Login};
        \node (auth) [decision, below of=start] {Credentials\\valides?};
        \node (role) [decision, below of=auth, yshift=-0.5cm] {Vérif.\\rôle};
        \node (admin) [process, below of=role, xshift=-3cm, yshift=-0.5cm] {Dashboard\\Admin};
        \node (recep) [process, below of=role, yshift=-0.5cm] {Dashboard\\Récep.};
        \node (client) [process, below of=role, xshift=3cm, yshift=-0.5cm] {Portail\\Client};
        \node (error) [process, right of=auth, xshift=2cm] {Erreur};
        
        \draw [arrow] (start) -- (auth);
        \draw [arrow] (auth) -- node {Oui} (role);
        \draw [arrow] (auth) -- node {Non} (error);
        \draw [arrow] (role) -- node {Admin} (admin);
        \draw [arrow] (role) -- node {Récep.} (recep);
        \draw [arrow] (role) -- node {Client} (client);
    \end{tikzpicture}
    \caption{Flux d'authentification et routage par rôle}
\end{figure}

\section{Algorithmes Clés}

\subsection{Validation des Disponibilités}

L'algorithme de vérification des disponibilités est crucial pour éviter les conflits de réservation :

\begin{lstlisting}[caption={Pseudo-code de vérification des disponibilités}]
fonction verifier_disponibilite(chambre, date_debut, date_fin):
    pour chaque reservation dans reservations:
        si reservation.chambre == chambre ET
           reservation.statut == "ACTIVE" ET
           dates_se_chevauchent(reservation, date_debut, date_fin):
            retourner INDISPONIBLE
    retourner DISPONIBLE
\end{lstlisting}

\subsection{Calcul Précis des Nuitées}

Nous utilisons les fonctions \texttt{mktime()} et \texttt{difftime()} de la bibliothèque standard C pour un calcul précis :

\begin{lstlisting}[caption={Calcul des nuitées avec mktime}]
int calculer_nuitees(const char* date_debut, const char* date_fin) {
    struct tm tm_debut = {0}, tm_fin = {0};
    
    // Parsing des dates
    sscanf(date_debut, "%d-%d-%d", 
           &tm_debut.tm_year, &tm_debut.tm_mon, &tm_debut.tm_mday);
    sscanf(date_fin, "%d-%d-%d",
           &tm_fin.tm_year, &tm_fin.tm_mon, &tm_fin.tm_mday);
    
    // Ajustement format struct tm
    tm_debut.tm_year -= 1900;
    tm_debut.tm_mon -= 1;
    tm_fin.tm_year -= 1900;
    tm_fin.tm_mon -= 1;
    
    // Conversion et calcul
    time_t time_debut = mktime(&tm_debut);
    time_t time_fin = mktime(&tm_fin);
    double diff_secondes = difftime(time_fin, time_debut);
    
    return (int)(diff_secondes / 86400); // 86400 sec par jour
}
\end{lstlisting}

% ========== CHAPITRE 3 : RÉALISATION TECHNIQUE ==========
\chapter{Réalisation Technique}

\section{Organisation du Code Source}

\subsection{Structure des Répertoires}

\begin{lstlisting}[language={}, basicstyle=\small\ttfamily, frame=none, numbers=none]
hotel_reservation_system/
|-- src/                  # Logique metier
|   |-- clients.c
|   |-- chambres.c
|   |-- reservations.c
|   |-- fichiers.c
|   |-- auth.c
|   +-- data_init.c
|-- ui/                   # Interface utilisateur
|   |-- ui.c
|   |-- ui_draw.c
|   |-- ui_theme.c
|   |-- ui_login.c
|   |-- ui_client_portal.c
|   +-- ...
|-- include/              # En-tetes
|   |-- structures.h
|   |-- clients.h
|   |-- chambres.h
|   +-- ...
|-- data/                 # Fichiers de donnees
+-- Makefile
\end{lstlisting}

\subsection{Compilation et Dépendances}

Le projet utilise \texttt{make} pour automatiser la compilation :

\begin{lstlisting}[language=make, caption={Extrait du Makefile}]
CC = gcc
CFLAGS = -Wall -Wextra -Iinclude
LDFLAGS = -lncursesw

SOURCES = $(wildcard src/*.c ui/*.c)
OBJECTS = $(SOURCES:.c=.o)

hotel_app.exe: $(OBJECTS) main.o
    $(CC) $(CFLAGS) -o $@ $^ $(LDFLAGS)

clean:
    rm -f $(OBJECTS) main.o hotel_app.exe
\end{lstlisting}

\section{Modules Fonctionnels}

\subsection{Module de Gestion des Clients}

Le module \texttt{clients.c} implémente les opérations CRUD :

\begin{itemize}
    \item \texttt{ajouter\_client()} : Ajout avec génération automatique d'ID
    \item \texttt{modifier\_client()} : Mise à jour des informations
    \item \texttt{supprimer\_client()} : Suppression avec vérification des contraintes
    \item \texttt{rechercher\_client()} : Recherche par nom, ID ou email
\end{itemize}

\subsection{Module de Gestion des Réservations}

Le module \texttt{reservations.c} gère le cycle de vie complet :

\begin{itemize}
    \item Validation des dates (pas de dates passées, début < fin)
    \item Vérification des disponibilités en temps réel
    \item Calcul automatique du montant basé sur les nuitées
    \item Gestion des statuts (ACTIVE, ANNULEE)
    \item Soft delete pour conservation de l'historique
\end{itemize}

\subsection{Module d'Interface Utilisateur}

L'interface ncurses offre une expérience professionnelle :

\begin{itemize}
    \item \textbf{Système de couleurs} : 8 paires de couleurs définies dans \texttt{ui\_theme.c}
    \item \textbf{Formulaires wizards} : Navigation Tab/Shift-Tab entre champs
    \item \textbf{Validation en temps réel} : Retour visuel immédiat
    \item \textbf{Composants réutilisables} : Boîtes de dialogue, listes scrollables
\end{itemize}

\section{Gestion de la Mémoire}

\subsection{Stratégie d'Allocation}

Nous utilisons une approche hybride :

\begin{itemize}
    \item \textbf{Tableaux statiques} : Pour les limites connues (MAX\_CLIENTS = 100)
    \item \textbf{Allocation dynamique} : Pour les buffers temporaires
    \item \textbf{Libération systématique} : Chaque \texttt{malloc()} a son \texttt{free()}
\end{itemize}

\subsection{Prévention des Fuites Mémoire}

\begin{lstlisting}[caption={Exemple de gestion propre de la mémoire}]
char* buffer = malloc(256 * sizeof(char));
if (buffer == NULL) {
    // Gestion d'erreur
    return ERROR_MALLOC;
}

// Utilisation du buffer
strncpy(buffer, source, 255);
buffer[255] = '\0';

// Libération systématique
free(buffer);
buffer = NULL;
\end{lstlisting}

\section{Persistance des Données}

\subsection{Format Binaire}

Les données sont stockées en format binaire pour :

\begin{itemize}
    \item Rapidité de lecture/écriture (pas de parsing)
    \item Compacité (pas de métadonnées textuelles)
    \item Intégrité (typage fort conservé)
\end{itemize}

\subsection{Opérations de Fichier}

\begin{lstlisting}[caption={Sauvegarde des clients}]
void sauvegarder_clients(Client clients[], int nb_clients) {
    FILE* f = fopen("data/clients.dat", "wb");
    if (f == NULL) {
        perror("Erreur ouverture fichier");
        return;
    }
    
    fwrite(&nb_clients, sizeof(int), 1, f);
    fwrite(clients, sizeof(Client), nb_clients, f);
    
    fclose(f);
}
\end{lstlisting}

% ========== CHAPITRE 4 : VALIDATION ET TESTS ==========
\chapter{Validation et Interface Utilisateur}

\section{Système d'Authentification et Enregistrement}

\subsection{Écran de Connexion}

Le système démarre sur un écran de connexion professionnel offrant une expérience utilisateur sécurisée et intuitive.

\begin{figure}[H]
    \centering
    \includegraphics[width=0.75\textwidth]{screenshots/1.png}
    \caption{Écran principal de connexion - Saisie des identifiants}
\end{figure}

L'écran de login présente une boîte de dialogue "LOGIN" sollicitant le nom d'utilisateur et le mot de passe. Les options de navigation sont clairement indiquées : ESC pour quitter l'application, ENTER pour se connecter, et F2 pour accéder à l'écran d'enregistrement. Le mot de passe est automatiquement masqué pour garantir la confidentialité.

\begin{figure}[H]
    \centering
    \includegraphics[width=0.75\textwidth]{screenshots/4.png}
    \caption{Tentative de connexion administrative - Pré-remplissage du champ utilisateur}
\end{figure}

Cette capture montre une tentative de connexion avec un compte administrateur. Le champ "Username" est pré-rempli avec "admin", démontrant la capacité du système à mémoriser les dernières saisies pour faciliter l'expérience utilisateur.

\subsection{Processus d'Enregistrement}

Le système offre un processus d'inscription guidé permettant la création de nouveaux comptes utilisateurs avec attribution de rôles.

\begin{figure}[H]
    \centering
    \includegraphics[width=0.75\textwidth]{screenshots/2-1.png}
    \caption{Écran d'enregistrement - État initial avec focus sur le champ Username}
\end{figure}

L'écran d'enregistrement dans son état initial montre le focus (surligné en gris) sur le champ "Username", prêt à recevoir la saisie utilisateur. Le rôle par défaut est configuré sur \texttt{< client >}, mais peut être modifié selon les besoins.

\begin{figure}[H]
    \centering
    \includegraphics[width=0.75\textwidth]{screenshots/2-2.png}
    \caption{Formulaire d'enregistrement complété - Compte client}
\end{figure}

Cette capture présente le formulaire d'enregistrement entièrement rempli. L'utilisateur a saisi "client2" comme nom d'utilisateur, un mot de passe masqué, et le rôle sélectionné est "client". Le bouton [ SUBMIT ] est visible et prêt à être activé.

\begin{figure}[H]
    \centering
    \includegraphics[width=0.75\textwidth]{screenshots/2-3.png}
    \caption{Confirmation d'enregistrement réussi pour l'utilisateur client2}
\end{figure}

L'écran de confirmation après un enregistrement réussi affiche les détails du compte "client2" avec le message "Success! Press Any Key", validant la création du compte dans le système.

\begin{figure}[H]
    \centering
    \includegraphics[width=0.75\textwidth]{screenshots/2-4.png}
    \caption{Enregistrement d'un utilisateur administrateur - Sélection du rôle admin}
\end{figure}

Cette capture montre le formulaire d'enregistrement configuré pour un utilisateur administratif. Le nom d'utilisateur est défini sur "admin" et le sélecteur de rôle a été basculé sur \texttt{< admin >}, accordant les privilèges d'administration complets.

\section{Tableaux de Bord et Interfaces Utilisateur}

\subsection{Dashboard Administrateur}

Le tableau de bord administrateur fournit une vue d'ensemble complète des métriques hôtelières et de l'état du système.

\begin{figure}[H]
    \centering
    \includegraphics[width=0.85\textwidth]{screenshots/4-1.png}
    \caption{Tableau de bord administrateur - Vue d'ensemble des métriques hôtelières}
\end{figure}

Le dashboard principal de l'administrateur présente quatre cartes de synthèse essentielles :
\begin{itemize}
    \item \textbf{Total Clients} : Affiche "8" comptes actifs
    \item \textbf{Total Rooms} : Montre "5" chambres avec un taux d'occupation calculé de "100.0\%"
    \item \textbf{Reservations} : Indique "5" réservations actives
    \item \textbf{Total Revenue} : Actuellement à "0.00 EUR"
\end{itemize}

Une barre de progression visuelle en bas de l'écran illustre le "Overall Occupancy Rate" (taux d'occupation global), offrant une visualisation immédiate de la performance de l'hôtel.

\subsection{Dashboard Client}

Les clients disposent d'un portail dédié avec fonctionnalités en libre-service.

\begin{figure}[H]
    \centering
    \includegraphics[width=0.75\textwidth]{screenshots/3.png}
    \caption{Tableau de bord client - Avertissement de profil manquant}
\end{figure}

Cette capture présente le dashboard client immédiatement après connexion en tant que "client2". Un message d'avertissement s'affiche : "Warning: No Client Profile found for username 'client2'. Please ask Reception to create a profile...". Cela illustre la distinction entre l'authentification (compte existant) et le profil client (détails personnels), garantissant l'intégrité relationnelle du système.

\section{Gestion des Clients}

\subsection{Liste et Recherche de Clients}

Le module de gestion des clients offre des capacités complètes de visualisation et de recherche.

\begin{figure}[H]
    \centering
    \includegraphics[width=0.85\textwidth]{screenshots/4-2.png}
    \caption{Écran de gestion des clients - Liste complète des enregistrements}
\end{figure}

L'écran de gestion des clients présente tous les utilisateurs enregistrés dans un format tabulaire avec les colonnes : ID, Nom, Prénom, Email et Téléphone. On peut observer des entrées notables comme "TAHIRI ALAOUI Yazid" (ID 1) et "HAMDAOUI Nisrine" (ID 3). Le pied de page offre des options pour [A]jouter, [E]diter, [D]upprimer et [S]earcher des clients.

\begin{figure}[H]
    \centering
    \includegraphics[width=0.65\textwidth]{screenshots/3-5.png}
    \caption{Boîte de dialogue de recherche client - Saisie du critère}
\end{figure}

La boîte de recherche client invite l'utilisateur à entrer un nom pour trouver des enregistrements spécifiques. Dans cet exemple, "[yazid]" a été saisi comme terme de recherche.

\begin{figure}[H]
    \centering
    \includegraphics[width=0.8\textwidth]{screenshots/3-6.png}
    \caption{Résultats de recherche - Correspondances pour la requête "yazid"}
\end{figure}

La fenêtre des résultats de recherche affiche les correspondances pour la requête "yazid", listant trois résultats avec leurs IDs, Noms, Emails et Numéros de téléphone. Cette fonctionnalité permet une localisation rapide des clients dans une base de données volumineuse.

\subsection{Ajout de Nouveau Client}

L'interface d'ajout de client permet la saisie manuelle de nouvelles fiches.

\begin{figure}[H]
    \centering
    \includegraphics[width=0.75\textwidth]{screenshots/4-3.png}
    \caption{Formulaire d'ajout de nouveau client - Champs vierges}
\end{figure}

Le formulaire "AJOUTER NOUVEAU CLIENT" fournit des champs vides permettant à l'administrateur de saisir manuellement les informations d'un nouveau client : Nom, Prénom, Email et Téléphone. L'ID est généré automatiquement par le système.

\section{Gestion de l'Inventaire des Chambres}

\subsection{Vue d'Ensemble et Modification}

Le module de gestion des chambres permet le suivi complet de l'inventaire hôtelier.

\begin{figure}[H]
    \centering
    \includegraphics[width=0.85\textwidth]{screenshots/4-4.png}
    \caption{Gestion des chambres - Inventaire complet avec statuts}
\end{figure}

L'écran de gestion des chambres liste l'inventaire de l'hôtel avec les colonnes : Numéro, Type, Prix/Nuit et Disponibilité. Le pied de page résume l'inventaire : "Total: 5 chambres | Disponibles: 5 | Occupees: 0", offrant une vue instantanée de la capacité hôtelière.

\begin{figure}[H]
    \centering
    \includegraphics[width=0.75\textwidth]{screenshots/4-6.png}
    \caption{Interface de modification de chambre - Édition des paramètres}
\end{figure}

L'interface "Edit Room" permet à l'administrateur de modifier les détails d'une chambre existante. Le "Room Number" (25) est en lecture seule pour préserver l'intégrité référentielle. L'utilisateur peut basculer le Type (actuellement "Suite") avec les touches fléchées Gauche/Droite et modifier le Prix/Nuit (actuellement "251.00").

\subsection{Ajout de Chambres}

Le système permet l'expansion de l'inventaire via un formulaire dédié.

\begin{figure}[H]
    \centering
    \includegraphics[width=0.75\textwidth]{screenshots/4-5.png}
    \caption{Ajout de nouvelle chambre en mode Debug - Saisie des paramètres}
\end{figure}

L'écran "Add New Room" en mode Debug affiche un flag "DEBUG: WINDOW IS VISIBLE" et permet la saisie des paramètres pour un nouveau bien immobilier : Numéro de chambre, Type (par défaut Simple) et Prix/Nuit. Ce mode debug aide au développement et au dépannage.

\section{Système de Réservation}

\subsection{Navigation et Réservation Côté Client}

Les clients peuvent parcourir l'inventaire et effectuer des réservations en autonomie.

\begin{figure}[H]
    \centering
   \includegraphics[width=0.85\textwidth]{screenshots/3-1.png}
    \caption{Interface Browse \& Book Rooms - Sélection de chambre disponible}
\end{figure}

L'interface "Browse \& Book Rooms" présente un tableau listant les Numéros de chambre, Types, Prix et Disponibilité. La chambre 25 (une Suite) est actuellement surlignée en bleu, indiquant la sélection en cours pour une potentielle réservation.

\begin{figure}[H]
    \centering
    \includegraphics[width=0.75\textwidth]{screenshots/3-2.png}
    \caption{Assistant de réservation - Étape 1 : Saisie des dates et contexte}
\end{figure}

Le Reservation Wizard - Step 1 demande à l'utilisateur de saisir les "Dates \& Context" pour une réservation, spécifiquement la Date de début, Date de fin et le Type de chambre désiré (Simple/Double). Cette approche wizard guide l'utilisateur pas à pas.

\begin{figure}[H]
    \centering
    \includegraphics[width=0.85\textwidth]{screenshots/3-3.png}
    \caption{Assistant de réservation - Étape 2 : Sélection avec codage couleur}
\end{figure}

Le Reservation Wizard - Step 2 permet à l'utilisateur de sélectionner une chambre spécifique. Un codage couleur intuitif est utilisé : le texte Vert indique les chambres "Available" (disponibles), tandis que le texte Rouge indique les chambres "Occupied" (occupées), par exemple la chambre 21 est occupée.

\subsection{Workflow Administratif de Réservation}

Les administrateurs et réceptionnistes disposent d'outils avancés de gestion des réservations.

\begin{figure}[H]
    \centering
    \includegraphics[width=0.75\textwidth]{screenshots/4-8.png}
    \caption{Assistant de réservation (vue admin) - Étape 1 : Configuration initiale}
\end{figure}

Le Reservation Wizard - Step 1 côté administrateur présente l'écran initial pour créer une nouvelle réservation. Il invite l'utilisateur à entrer la Date de début, Date de fin et le Type de chambre préféré (Single/Double), avec une interface similaire au portail client mais dans un contexte administratif.

\begin{figure}[H]
    \centering
    \includegraphics[width=0.75\textwidth]{screenshots/3-4.png}
    \caption{Assistant de réservation - Étape 3 : Association avec le profil client}
\end{figure}

L'étape 3 "Client Link" connecte une réservation à un profil utilisateur. Cette étape offre deux options : saisir directement le Client ID ou appuyer sur 'F' pour rechercher un client par son nom dans la base de données, facilitant l'association correcte.

\begin{figure}[H]
    \centering
    \includegraphics[width=0.9\textwidth]{screenshots/4-7.png}
    \caption{Gestion des réservations - Liste complète avec détails et montants}
\end{figure}

L'écran "Reservations Management" affiche une liste exhaustive de toutes les réservations avec les colonnes : ID, Nom du client, Numéro de chambre, Date de début, Date de fin et Montant total. Une réservation notable apparaît pour le client "yazid12" (ID 4) pour la chambre 10, couvrant 2 ans (2024-2026) pour un total de "365000.00 EU". Le pied de page offre des options pour [A]pprouver ou [R]efuser des réservations, ainsi que les fonctions standard d'édition et d'annulation.

\section{Système de Facturation}

\subsection{Génération et Consultation de Factures}

Le module de facturation automatise le calcul et la génération de documents.

\begin{figure}[H]
    \centering
    \includegraphics[width=0.85\textwidth]{screenshots/4-9.png}
    \caption{Menu principal Billing \& Invoices - Structure de table vide}
\end{figure}

Le menu principal "Billing \& Invoices" présente une structure de table prête à afficher les enregistrements de factures avec les colonnes : ID, Client ID, Nuits, Prix/Nuit et Total. Les options au bas permettent de [C]réer une nouvelle facture ou [V]isualiser une facture existante.

\begin{figure}[H]
    \centering
    \includegraphics[width=0.8\textwidth]{screenshots/4-10.png}
    \caption{Vue détaillée d'une facture générée - Facture \#4}
\end{figure}

La vue "FACTURE" affiche les détails pour la "Facture \#: 4", émise au client "yazid12". La facture couvre la chambre 10 pour une période de 731 nuits (du 12/01/2024 au 12/01/2026), résultant en un total de 365,500.00 MAD. Une option spécifique [P] Print to File est disponible pour exporter ces données vers un fichier texte.

\subsection{Export et Archivage}

Le système permet l'exportation des factures pour archivage et impression.

\begin{figure}[H]
    \centering
    \includegraphics[width=0.75\textwidth]{screenshots/4-11.png}
    \caption{Explorateur Windows - Confirmation d'export de facture}
\end{figure}

Cette capture d'écran de l'Explorateur de fichiers Windows montre le répertoire du projet. Une flèche rouge pointe vers un fichier nouvellement généré nommé \texttt{facture\_4.txt}, confirmant que la fonction "Print to File" de l'écran précédent a bien créé une facture au format texte, garantissant la traçabilité et l'archivage des transactions.

\section{Tests de Validation}

\subsection{Scénarios de Test Fonctionnels}

Le système a été validé à travers des scénarios de test couvrant tous les cas d'usage critiques.

\begin{table}[H]
\centering
\rowcolors{2}{gray!10}{white}
\begin{tabular}{L{4cm}C{5.5cm}L{4.5cm}}
\toprule
\textbf{\color{ensahblue}Scénario} & \textbf{\color{ensahblue}Actions} & \textbf{\color{ensahblue}Résultat Attendu} \\
\midrule
Authentification Admin & Login avec identifiants admin/admin123 & Accès au dashboard complet avec toutes fonctionnalités \\
\addlinespace[0.15em]
Enregistrement Client & Remplir formulaire avec rôle client & Compte créé, message de succès affiché \\
\addlinespace[0.15em]
Ajout Client & Saisie formulaire client complet & Client enregistré avec ID auto-généré \\
\addlinespace[0.15em]
Réservation Valide & Dates valides, chambre disponible & Réservation créée, chambre marquée occupée \\
\addlinespace[0.15em]
Conflit de Réservation & Dates se chevauchant pour même chambre & Erreur détectée, réservation refusée \\
\addlinespace[0.15em]
Recherche Client & Recherche par nom partiel & Liste filtrée des correspondances \\
\addlinespace[0.15em]
Facturation & Sélection réservation active & Facture générée avec calculs corrects \\
\addlinespace[0.15em]
Export Facture & Option Print to File & Fichier texte créé dans répertoire \\
\addlinespace[0.15em]
Portail Client & Login avec compte client & Accès limité aux fonctions client uniquement \\
\bottomrule
\end{tabular}
\caption{Scénarios de tests fonctionnels validés}
\end{table}

\subsection{Validation des Contraintes Métier}

Toutes les contraintes métier et règles de gestion ont été rigoureusement testées et validées :

\begin{itemize}
    \item [\checkmark] \textbf{Unicité des identifiants} : IDs uniques pour clients, réservations et factures
    \item [\checkmark] \textbf{Validation des formats} : Dates au format YYYY-MM-DD, emails valides
    \item [\checkmark] \textbf{Contraintes temporelles} : Interdiction des dates passées pour nouvelles réservations
    \item [\checkmark] \textbf{Cohérence des dates} : Vérification date\_début < date\_fin
    \item [\checkmark] \textbf{Détection des conflits} : Algorithme de chevauchement empêchant double réservation
    \item [\checkmark] \textbf{Calculs précis} : Nuitées calculées avec mktime/difftime pour précision
    \item [\checkmark] \textbf{Persistance} : Données correctement sauvegardées et récupérées entre sessions
    \item [\checkmark] \textbf{Intégrité référentielle} : Liens cohérents entre clients et réservations
    \item [\checkmark] \textbf{Contrôle d'accès} : RBAC respecté selon les rôles définis
\end{itemize}

% ========== CONCLUSION ==========
\chapter*{Conclusion}
\addcontentsline{toc}{chapter}{Conclusion}

\section*{Bilan du Projet}

La réalisation de ce système de gestion hôtelière a constitué une expérience formatrice à plusieurs niveaux. Sur le plan technique, nous avons consolidé notre maîtrise du langage C, en particulier dans ses aspects avancés : manipulation de pointeurs, gestion dynamique de la mémoire, programmation modulaire et utilisation de bibliothèques système.

L'intégration de la bibliothèque ncurses nous a permis de découvrir les subtilités de la programmation d'interfaces textuelles professionnelles, un domaine souvent négligé au profit des interfaces graphiques modernes, mais qui reste pertinent pour les systèmes embarqués et les environnements serveur.

\section*{Défis Rencontrés}

Plusieurs obstacles techniques ont jalonné le développement :

\begin{itemize}
    \item \textbf{Gestion des dates} : L'implémentation d'un calcul précis des nuitées a nécessité l'utilisation de \texttt{mktime()} et \texttt{difftime()}, après avoir constaté les limitations des approches arithmétiques naïves.
    \item \textbf{Contrôle ncurses} : La maîtrise des modes de saisie (notamment le mode \texttt{cbreak} et la désactivation de l'écho) a demandé une compréhension approfondie de la documentation.
    \item \textbf{Persistance binaire} : La gestion robuste des erreurs lors des opérations de fichiers a été essentielle pour garantir l'intégrité des données.
    \item \textbf{Architecture RBAC} : L'implémentation du contrôle d'accès basé sur les rôles a nécessité une refonte partielle de la structure de navigation.
\end{itemize}

\section*{Acquis Techniques}

Ce projet nous a permis de développer des compétences solides en :

\begin{itemize}
    \item Conception d'architecture logicielle multicouche
    \item Modélisation de données et gestion des relations entre entités
    \item Implémentation d'algorithmes de validation et de recherche
    \item Gestion rigoureuse de la mémoire en C
    \item Développement d'interfaces utilisateur avec ncurses
    \item Utilisation d'outils de build (Make)
    \item Versionnement de code (Git)
\end{itemize}

\section*{Perspectives d'Évolution}

Plusieurs axes d'amélioration sont envisageables :

\subsection*{Fonctionnalités}

\begin{itemize}
    \item \textbf{Statistiques avancées} : Taux d'occupation, revenus mensuels, clients fidèles
    \item \textbf{Export multi-formats} : PDF, CSV pour les rapports
    \item \textbf{Notifications} : Alertes pour les réservations imminentes
    \item \textbf{Gestion du personnel} : Attribution des chambres au personnel de ménage
    \item \textbf{Tarification dynamique} : Modulation des prix selon la saison
\end{itemize}

\subsection*{Techniques}

\begin{itemize}
    \item \textbf{Base de données} : Migration vers SQLite pour une meilleure scalabilité
    \item \textbf{Cryptage} : Hashage des mots de passe (bcrypt)
    \item \textbf{Logs d'audit} : Traçabilité complète des opérations
    \item \textbf{Tests unitaires} : Couverture avec framework de test C
    \item \textbf{Documentation} : Génération automatique avec Doxygen
\end{itemize}

\section*{Conclusion Finale}

Ce projet concrétise notre progression en programmation système et démontre qu'une application C bien architecturée peut rivaliser en ergonomie et en fonctionnalités avec des solutions développées dans des langages de plus haut niveau. Il confirme également que la rigueur imposée par le langage C, loin d'être une contrainte, constitue une école d'excellence pour tout développeur souhaitant comprendre en profondeur le fonctionnement des systèmes informatiques.

Nous sommes fiers du résultat obtenu et convaincus que les compétences acquises lors de ce projet constitueront un socle solide pour nos futurs développements, qu'ils soient en C ou dans d'autres langages.

\vspace{2cm}
\begin{flushright}
\textit{Yazid TAHRI ALAOUI, Dounia LAMDAKKI, Kaoutar BENOUAHI}\\
\textit{ENSAH - ID1 - 2024/2025}
\end{flushright}

\end{document}
